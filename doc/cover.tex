% -*-latex-*-
% 
% For questions, comments, concerns or complaints:
% thesis@mit.edu
% 
%
% $Log: cover.tex,v $
% Revision 1.8  2008/05/13 15:02:15  jdreed
% Degree month is June, not May.  Added note about prevdegrees.
% Arthur Smith's title updated
%
% Revision 1.7  2001/02/08 18:53:16  boojum
% changed some \newpages to \cleardoublepages
%
% Revision 1.6  1999/10/21 14:49:31  boojum
% changed comment referring to documentstyle
%
% Revision 1.5  1999/10/21 14:39:04  boojum
% *** empty log message ***
%
% Revision 1.4  1997/04/18  17:54:10  othomas
% added page numbers on abstract and cover, and made 1 abstract
% page the default rather than 2.  (anne hunter tells me this
% is the new institute standard.)
%
% Revision 1.4  1997/04/18  17:54:10  othomas
% added page numbers on abstract and cover, and made 1 abstract
% page the default rather than 2.  (anne hunter tells me this
% is the new institute standard.)
%
% Revision 1.3  93/05/17  17:06:29  starflt
% Added acknowledgements section (suggested by tompalka)
% 
% Revision 1.2  92/04/22  13:13:13  epeisach
% Fixes for 1991 course 6 requirements
% Phrase "and to grant others the right to do so" has been added to 
% permission clause
% Second copy of abstract is not counted as separate pages so numbering works
% out
% 
% Revision 1.1  92/04/22  13:08:20  epeisach

% NOTE:
% These templates make an effort to conform to the MIT Thesis specifications,
% however the specifications can change.  We recommend that you verify the
% layout of your title page with your thesis advisor and/or the MIT 
% Libraries before printing your final copy.
\title{Trend or No Trend: A Novel Nonparametric Method for Classifying Time Series}

\author{Stanislav Nikolov}
% If you wish to list your previous degrees on the cover page, use the 
% previous degrees command:
%       \prevdegrees{A.A., Harvard University (1985)}
% You can use the \\ command to list multiple previous degrees
%       \prevdegrees{B.S., University of California (1978) \\
%                    S.M., Massachusetts Institute of Technology (1981)}

\prevdegrees{S.B., Massachusetts Institute of Technology (2011)}

\department{Department of Electrical Engineering and Computer Science}

% If the thesis is for two degrees simultaneously, list them both
% separated by \and like this:
% \degree{Doctor of Philosophy \and Master of Science}
\degree{Master of Engineering in Electrical Engineering and Computer Science}

% As of the 2007-08 academic year, valid degree months are September, 
% February, or June.  The default is June.
\degreemonth{September}
\degreeyear{2012}
\thesisdate{August 15, 2012}

%% By default, the thesis will be copyrighted to MIT.  If you need to copyright
%% the thesis to yourself, just specify the `vi' documentclass option.  If for
%% some reason you want to exactly specify the copyright notice text, you can
%% use the \copyrightnoticetext command.  
%\copyrightnoticetext{\copyright IBM, 1990.  Do not open till Xmas.}

% If there is more than one supervisor, use the \supervisor command
% once for each.
\supervisor{Prof. Devavrat Shah}{Jamieson Career Development Associate Professor of Electrical Engineering and Computer Science}

\cosupervisor{Dr. Satanjeev Banerjee}{Engineer, Twitter Inc.}

% This is the department committee chairman, not the thesis committee
% chairman.  You should replace this with your Department's Committee
% Chairman.
\chairman{Prof. Dennis M. Freeman}{Chairman, Masters of Engineering Thesis Committee}

% Make the titlepage based on the above information.  If you need
% something special and can't use the standard form, you can specify
% the exact text of the titlepage yourself.  Put it in a titlepage
% environment and leave blank lines where you want vertical space.
% The spaces will be adjusted to fill the entire page.  The dotted
% lines for the signatures are made with the \signature command.
\maketitle

% The abstractpage environment sets up everything on the page except
% the text itself.  The title and other header material are put at the
% top of the page, and the supervisors are listed at the bottom.  A
% new page is begun both before and after.  Of course, an abstract may
% be more than one page itself.  If you need more control over the
% format of the page, you can use the abstract environment, which puts
% the word "Abstract" at the beginning and single spaces its text.

%% You can either \input (*not* \include) your abstract file, or you can put
%% the text of the abstract directly between the \begin{abstractpage} and
%% \end{abstractpage} commands.

% First copy: start a new page, and save the page number.
\cleardoublepage
% Uncomment the next line if you do NOT want a page number on your
% abstract and acknowledgments pages.
% \pagestyle{empty}
\setcounter{savepage}{\thepage}
\begin{abstractpage}
%% The text of your abstract and nothing else (other than comments) goes here.
%% It will be single-spaced and the rest of the text that is supposed to go on
%% the abstract page will be generated by the abstractpage environment.  This
%% file should be \input (not \include 'd) from cover.tex.

In supervised classification, one attempts to learn a model of how objects map
to labels. This involves selecting the best model from some model space,
prefering a model that fits the data but has low complexity. The choice of model
space encodes assumptions about the problem, for example, via a kernel and its
associated Reproducing Kernel Hilbert Space. We propose a different setting for
model specification and selection in supervised learning based on a {\em latent
  source model}. In this setting, the model is specified by a small collection
of unknown {\em latent sources}. We posit that the data were generated by these
latent sources and that there is a stochastic model relating latent sources and
observations.

With this setting in mind, we propose a classification method that avoids
searching over the model space, and is in fact, entirely unaware of what the
latent sources are or how many there are. Instead, our method relies on large
amounts of data as a proxy for the unknown latent sources. We perform
classification by directly computing the conditional class probabilities for an
observation based on our stochastic model. This approach results in a surprising
and natural interpretation --- that to see how likely it is that an observation
belongs to a certain class, we can simply observe how much it resembles other
examples of that class. This is well-suited to problems with large amounts of
labeled data.

We extend this approach to the problem of online timeseries classification. In
the binary case, we derive a maximum likelihood estimator for online signal
detection and an associated implementation that is simple, efficient, and
scalable. We demonstrate the merit of our approach by applying it to the task of
detecting {\em trending topics} on Twitter, and show that in many cases, we can
detect trending topics before they are identified by Twitter while maintaining a
low rate of error.

%(TODO: how is it ML?) 

%For example, in Tikhonov Regularization and associated special cases such as
%SVMs or Regularized Least Squares, the model space is specified by the choice of
%a kernel, which encodes similarities between data points, and a corresponding
%Reproducing Kernel Hilbert Space from which the classification function is
%chosen.

%We propose that belonging to a particular class amounts to having been generated
%by the same source as some labeled example belonging to that class. 

\end{abstractpage}

% Additional copy: start a new page, and reset the page number.  This way,
% the second copy of the abstract is not counted as separate pages.
% Uncomment the next 6 lines if you need two copies of the abstract
% page.
% \setcounter{page}{\thesavepage}
% \begin{abstractpage}
% %% The text of your abstract and nothing else (other than comments) goes here.
%% It will be single-spaced and the rest of the text that is supposed to go on
%% the abstract page will be generated by the abstractpage environment.  This
%% file should be \input (not \include 'd) from cover.tex.

In supervised classification, one attempts to learn a model of how objects map
to labels. This involves selecting the best model from some model space,
prefering a model that fits the data but has low complexity. The choice of model
space encodes assumptions about the problem, for example, via a kernel and its
associated Reproducing Kernel Hilbert Space. We propose a different setting for
model specification and selection in supervised learning based on a {\em latent
  source model}. In this setting, the model is specified by a small collection
of unknown {\em latent sources}. We posit that the data were generated by these
latent sources and that there is a stochastic model relating latent sources and
observations.

With this setting in mind, we propose a classification method that avoids
searching over the model space, and is in fact, entirely unaware of what the
latent sources are or how many there are. Instead, our method relies on large
amounts of data as a proxy for the unknown latent sources. We perform
classification by directly computing the conditional class probabilities for an
observation based on our stochastic model. This approach results in a surprising
and natural interpretation --- that to see how likely it is that an observation
belongs to a certain class, we can simply observe how much it resembles other
examples of that class. This is well-suited to problems with large amounts of
labeled data.

We extend this approach to the problem of online timeseries classification. In
the binary case, we derive a maximum likelihood estimator for online signal
detection and an associated implementation that is simple, efficient, and
scalable. We demonstrate the merit of our approach by applying it to the task of
detecting {\em trending topics} on Twitter, and show that in many cases, we can
detect trending topics before they are identified by Twitter while maintaining a
low rate of error.

%(TODO: how is it ML?) 

%For example, in Tikhonov Regularization and associated special cases such as
%SVMs or Regularized Least Squares, the model space is specified by the choice of
%a kernel, which encodes similarities between data points, and a corresponding
%Reproducing Kernel Hilbert Space from which the classification function is
%chosen.

%We propose that belonging to a particular class amounts to having been generated
%by the same source as some labeled example belonging to that class. 

% \end{abstractpage}

\cleardoublepage

\section*{Acknowledgments}

This is the acknowledgements section.  You should replace this with your
own acknowledgements.

%%%%%%%%%%%%%%%%%%%%%%%%%%%%%%%%%%%%%%%%%%%%%%%%%%%%%%%%%%%%%%%%%%%%%%
% -*-latex-*-
