%% This is an example first chapter.  You should put chapter/appendix that you
%% write into a separate file, and add a line \include{yourfilename} to
%% main.tex, where `yourfilename.tex' is the name of the chapter/appendix file.
%% You can process specific files by typing their names in at the 
%% \files=
%% prompt when you run the file main.tex through LaTeX.
\chapter{Introduction}
\section{Motivation}
Detection, classification, and prediction of events in temporal streams of information are ubiquitous problems in science, engineering and society (TOO BROAD?). From detecting malfunctions in a production plant, to predicting an imminent market crash, to revealing emerging popular topics in a social network, extracting useful information from time-varying data is fundamental for understanding the processes around us and making decisions.

(The method does not require that the indpendent variable is time.)

In recent years, there has been an explosion in the availability of data related to virtually every human endeavor --- data that demands to be analyzed and turned into valuable insights. Massive streams of user generated documents, such as blogs and tweets, as well as data from portable electronic devices, provide an amazing opportunity to study the dynamics of human social interaction online and face to face (cite Pentland?). How do people make decisions? Who are they influenced by? How do ideas and behaviors spread and evolve? These are questions that have been impossible to study empirically at scale until recent times. In healthcare, records of over-the-counter medication sales (cite National Retail Data Monitor) as well as search engine queries can anticipate the outbreak of disease and provide insight into the most effective ways to limit its spread. Particle collision experiments at the Large Hadron Collider generate more than 15 petabytes (SOURCE?) of data every year that promises to reveal the most fundamental physical truths.

(Does the physics example really fit in here? It's not a network example, but it looks like we won't be using anything network specific in the method.)
%In hospitals, continuous patient monitoring can help identify problematic situations early on and allow medical professionals to make life-saving decisions. 

%The recent financial meltdown (ugh cliche) highlights the need for increased efforts ??? what about Black Swan events, etc... of course there are things we can't predict, but there should be plenty that we can that are important... what's a good example?

At the same time, advances in distributed computing have made it easier than ever to exploit the structure in that data to do inference at scale. ELABORATE?

All of the examples mentioned above share a common setting. There is an underlying process whose observable properties generate timeseries. We would like to observe those timeseries to 
\begin{itemize}
\item detect anomalous events
\item classify the current activity of the timeseries into event types
\item predict the values of the timeseries at some future point.
\end{itemize}

This is difficult to do in general. Many real-world processes defy simple models that describe their behavior. A quote from ``The Unreasonable Effectiveness of Data'' by Halevy, Norvig, and Pereira sums it up:
\begin{quote}
``{\em Eugene Wigner's article `The Unreasonable Effectiveness of Mathematics in the Natural Sciences' examines why so much of physics can be neatly explained with simple mathematical formulas such as $f = ma$ or $e = mc^2$. Meanwhile, sciences that involve human beings rather than elementary particles have proven more resistant to elegant mathematics. Economists suffer from physics envy over their inability to neatly model human behavior. An informal, incomplete grammar of the English language runs over 1,700 pages. Perhaps when it comes to natural language processing and related fields, we're doomed to complex theories that will never have the elegance of physics equations. But if that's so, we should stop acting as if our goal is to author extremely elegant theories, and instead embrace complexity and make use of the best ally we have: the unreasonable effectiveness of data.}''
\end{quote}

Like language, the behavior of complex systems rarely admits a simple model that works well in practice. Like machine translation and speech recognition, there is an ever growing amount of data ``in the wild'' about processes like epidemics, rumor-spreading in social networks, financial transactions, and more. The inadequacy of simple models for complex behavior requires an approach that embraces this wealth of data and it highlights the need for a unified framework that efficiently exploits the structure in that data to do detection, classification, and prediction in timeseries.

\section{Previous Work}

Event detection in timeseries.

Classification in timeseries.

Prediction in timeseries.

Models with theoretical justification for the underlying model of the process. Network cascade models. Branching process models.
Epidemics, Social Network rumor spreading

(Nonparametric) timeseries clustering methods

Explicit generative model? Mixture model?

\section{My Approach}
Simple models prove ineffective when exposed to the vast variety of real world situations. [SAY MORE ABOUT THIS] I propose a nonparametric framework for doing inference on timeseries (VAGUE). In this model, we posit that there exist of a set of latent timeseries, each corresponding to a prototypical event of a certain type, and that each observed timeseries is a noisy observation of one of the latent timeseries.

\subsection{Anomaly Detection}
A timeseires in a given window is compared to a set of reference timeseries that represents ``normal'' events. We compute the probability that the observed timeseries was generated by a latent timeseries corresponding to a timeseries in the reference set. We then declare observations with low probability to be anomalies.

\subsection{Classification}
A timeseries in a given window is compared to two {\em reference} sets of timeseries --- one consisting of positive examples and the other of negative examples. We want to find out whether it is more likely that the observed timeseries was generated by one of the latent timeseries in the positive reference set or one of the latent timeseries in the negative reference set.

\subsection{Prediction} A timeseries in a given window is compared to a single reference set of timeseries. We want to find out the probability that the observed timeseries was generated from the same latent timeseries as each of the timeseries in the set. We then compute the most likely change in the observed timeseries by observing the changes in each reference timeseries and weighting the change by the probability that the observed timeseries and the reference timeseries were generated by the same latent timeseries.
 
************Lots of work in clustering, but this has an explicit probabilistic model.

\subsection{Application: Detecting Outbreaks of Popular Topics on Twitter}
As an application, I apply the method (CALL IT SOMETHING) to the problem of detecting emerging popular topics (called {\em trends}) on Twitter. I use a set of labeled examples of timeseries correspondingn to topics that eventually became trending topics and topics that did not. I then perform online classification of a given topic to label it as trending or not trending at a particular time.
***********Talk about results and main contributions.
