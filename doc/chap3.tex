\chapter{Algorithm}
\label{ch:alg}

%TODO: Graphics of shift invariant distance, closest match, etc.

%TODO. chapter references
In this chapter, I describe the practical implementation of the method described
in Chapter \ref{ch:method}.

% Assume we have the normalize and sliced signals already and just talk about
% detection.
\section{Overview}

The goal of the algorithm is to perform online classification of an infinite
stream of samples from an observed digital signal. We will focus on the case of
binary classification, in which we have positive signals and negative signals,
but the results can be extended to multiple classes.  For binary classification,
one could imagine that one class represents events and the other non-events, and
that we would like to detect events as soon as they happen.

To predict which class the observed signal belongs to at a given point in time,
we compute the probability that the recent samples of the observed signal were
generated by a ``positive'' process and the probability that they were generate
by a ``negative'' process, based on previously observed positive and negative
{\em reference signals}. Recall from Chapter \ref{ch:method} that a signal is
generated by a particular type of process if it shares the same latent process
as a signal of that type. To compute the probability that the recently observed
samples share the same process with a particular reference signal, we compute
the distance between the trajectory consisting of the recently observed samples
and all trajectories of the same size in the reference signal, and take the
minimum over all such trajectories. 

\section{Implementation}
In practice, the computation of probabilities amounts to nothing more than
computing distances. To compute the probability that an observation belongs to a
particular class, one simply computes the distance from the observation to each
reference signal in that class in order to see how much the observation
resembles the reference signals for that class.

Algorithm \ref{alg:Detect} contains the core detection logic. 
% TODO: Justification for shift-invariance.
\begin{algorithm}
\caption{Perform online binary classification on the infinite stream $\mb
  s_{\infty}$ using sets of positive and negative reference signals $R_+$ and
  $R_-$.}
\label{alg:Detect}
\at{Detect}($\mb s_{\infty}$, $\mathcal{R}_+$, $\mathcal{R}_-$, $\gamma$, $\theta$,
$D_{req}$):
\begin{algorithmic}[1]
\STATE \vt{ConsecutiveDetections} $\leftarrow$ 0
\LOOP
  \STATE $\mb s$ $\leftarrow$ \at{UpdateObservation}($\mb s_{\infty}$, $N_{obs}$)
  \FOR{$\mb r$ {\bf in} $\mathcal{R}_+$}
    \STATE \vt{PosDists}.\at{Append}(\at{DistToReference}($\mb s$, $\mb r$))
  \ENDFOR
  \FOR{$\mb r$ {\bf in} $\mathcal{R}_-$}
    \STATE \vt{NegDists}.\at{Append}(\at{DistToReference}($\mb s$, $\mb r$))
  \ENDFOR
  \STATE \vt{R} = \at{ProbClass}(\vt{PosDists}, $\gamma$) / \at{ProbClass}(\vt{NegDists}, $\gamma$)
  \IF{\vt{R} $> \theta$ }
    \IF{\vt{ConsecutiveDetections} $>$ $D_{req}$}
      \STATE \vt{DetectionTime} $\leftarrow$ \at{CurrentTime}()
      \RETURN \vt{DetectionTime}
    \ELSE
      \STATE \vt{ConsecutiveDetections} $\leftarrow$ \vt{ConsecutiveDetections} + 1
    \ENDIF
  \ELSE
    \STATE \vt{ConsecutiveDetections} $\leftarrow$ 0
  \ENDIF
\ENDLOOP
\end{algorithmic}
\end{algorithm}
At each time step, it updates the observation $\mb
s$ so that $\mb s$ contains the latest $N_{obs}$ samples from the infinite
stream $\mb s_{\infty}$ and computes the distances \vt{PosDists}
(resp. \vt{NegDists}) to reference signals of the positive class (resp. negative
class). A detection is declared whenever the ratio of class probabilities
$R(\mb s)$ exceeds the threshold $\theta$ for $D_{req}$ consecutive time steps.

Algorithm \ref{alg:DistToReference} computes the distance between a reference
signal $\mb r$ and an observation $\mb s$. 
\begin{algorithm}
\caption{Compute the minimum distance between $\mb s$ and all pieces of $\mb r$
  of the same length as $\mb s$.}
\label{alg:DistToReference}
\at{DistToReference}($\mb s$, $\mb r$):
\begin{algorithmic}[1]
  \STATE $N_{obs}$ $\leftarrow$ \at{length}($\mb s$)
  \STATE $N_{ref}$ $\leftarrow$ \at{length}($\mb r$)
  \STATE \vt{MinDist} = $\infty$
  \FOR{$i=1$ \TO $N_{ref} - N_{obs} + 1$}
    \STATE \vt{MinDist} = \at{Min}(\vt{MinDist}, \at{Dist}($\mb r_{i:i+N_{obs}-1}$, $\mb s$))
  \ENDFOR
  \RETURN \vt{MinDist}
\end{algorithmic}
\end{algorithm}
Since the reference signal is generally longer than the observation, we compute
the minimum distance (Algorithm \ref{alg:DistToSignal}) across all pieces of
$\mb r$ of the same size as $\mb s$.

Algorithm \ref{alg:DistToSignal} simply computes the Euclidean distance between
two signals of the same size.
\begin{algorithm}
\caption{Compute the distance between two signals $\mb s$ and $\mb t$ of the same
    length}
\label{alg:DistToSignal}
\at{Dist}($\mb s$, $\mb t$):
\begin{algorithmic}[1]
\STATE \vt{D} $\leftarrow$ 0
\FOR{$i=1$ to \at{length}($\mb s$)}
  \STATE \vt{D} $\leftarrow$ \vt{D} + $(s_i - t_i)^2$
\ENDFOR
\RETURN \vt{D}
\end{algorithmic}
\end{algorithm}

Using the distances from an observation to the reference signals of a class, we
compute a number proportional the probability that the observation belongs to
the class (Algorithm \ref{alg:ProbClass}).
\begin{algorithm}
\caption{Using the distances of an observation to the reference signals of a
  certain class, compute a number proportional to the probability that the
  observation belongs to that class.}
\label{alg:ProbClass}
\at{ProbClass}(\vt{Dists}, $\gamma$):
\begin{algorithmic}[1]
\STATE $P \leftarrow 0$
\FOR{$i=1$ to \at{Length}(\vt{Dists})}
  \STATE $P \leftarrow P + \exp\left(-\gamma Dists_i\right)$
\ENDFOR
\RETURN $P$
\end{algorithmic}
\end{algorithm}

\clearpage
\section{Performance and Scalability}
% TODO
% Summarize performance
To do detection on an infinite stream for $T$ time steps, with $|\mathcal{R}_+|$
positive reference signals and $|\mathcal{R}_-|$ negative reference signals of
length $N_{ref}$, and observations of length $N_{obs}$, our rudimentary
implementation runs in $\mathcal{O}(TN_{ref}(|\mathcal{R}_-| +
|\mathcal{R}_-|))$ time. In practice, the algorithm can be made faster by a
constant factor by not performing detection on every time step, not computing
distances based on the full $N_{obs}$ samples, or not comparing the observation
to every single slice of the reference signal.

Clearly, the computational cost of our implementation grows with the amount of
data. Nevertheless, our approach is scalable, since one can compute in parallel
the scores for each of the topics, as well as each of the reference signal
distances for each topic.

% Potential to be Super fast using trajectory indexing and retrieval
A more sophisticated version of our algorithm would use an approach based on
time series indexing. For instance, Rakthanmanon et al. have shown a way to
efficiently search over trillions of time series subsequences
\cite{Rakthanmanon}. Since our probability-based metric involves exponential
decay based on the distance between signals, most reference signals that are far
away from the observation can safely be ignored. Thus, instead of computing the
distance to all reference signals, which could become costly, we can operate on
only a very small fraction of them without significantly affecting the outcome.
